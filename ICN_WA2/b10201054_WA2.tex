\documentclass[12pt,a4paper]{article}
\usepackage{graphicx}
\usepackage{amsmath}
\usepackage{bm}
\usepackage{interval}
\usepackage{amssymb}
\usepackage{makecell}
\usepackage[letterpaper,top=2cm,bottom=2cm,left=3cm,right=3cm,marginparwidth=1.75cm]{geometry}
\usepackage[colorlinks=true, allcolors=blue]{hyperref}

\title{ICN Written Assignment 2}
\author{Student ID: b10201054}

\begin{document}
\maketitle

\section*{Problem 1}

\begin{enumerate}
    \item [(a)] 
        The fowarding table is:
        \begin{table}[!ht]
            \centering
            \begin{tabular}{|l|l|}
            \hline
                Destination Address Range & Link Interface \\ \hline
                11100000 00 & 0 \\ \hline
                11100000 01000000 & 1 \\ \hline
                11100000  & 2 \\ \hline
                11100001 0 & 2 \\ \hline
                otherwise & 3 \\ \hline
            \end{tabular}
            \caption{Forwarding Table}
        \end{table}
        
    \item[(b)]
        
        11001000 10010001 01010001 01010101 doesn't match any prefix in destination address range, so the packet would be sent to Link 3.

        11100001 01000000 11000011 00111100 matches matches the prefix of Link 1 longer than Link 2, so the packet would be sent to Link 1.

        11100001 10000000 00010001 01110111 doesn't matches any prefix in destination address range, so the packet would be sent to Link 3.
        
\end{enumerate}

\section*{Problem 2}    

\begin{enumerate}
    \item [(a)] 
        214.97.254/23 to 32-bit binary is 11010110 01100001 11111110 ********.\\
        The first 23 bits is the subnet part, so we can only use the last 9 digits.\\
        Subnet A: 214.97.255/24
        Subnet B: 214.97.254.0/25 - 214.97.254.0/29 \\
        Subnet C: 214.97.254.128/25 \\
        Subnet D: 214.97.254.0/31 \\
        Subnet E: 214.97.254.2/31 \\
        Subnet F: 214.97.254.4/30
    \newpage
    \item[(b)]
        Forwarding table:
        \begin{table}[!ht]
            \centering
            \begin{tabular}{|l|l|}
            \hline
                Longest Prefix Match & Outgoing Interface \\ \hline
                11010110 01100001 11111111 & A \\ \hline
                11010110 01100001 11111110 0000000 & D \\ \hline
                11010110 01100001 11111110 000001  & F \\ \hline
            \end{tabular}
            \caption{Router 1 Forwarding Table}
        \end{table}

        \begin{table}[!ht]
            \centering
            \begin{tabular}{|l|l|}
            \hline
                Longest Prefix Match & Outgoing Interface \\ \hline
                11010110 01100001 11111110 1 & C \\ \hline
                11010110 01100001 11111110 0000001 & E \\ \hline
                11010110 01100001 11111110 000001  & F \\ \hline
            \end{tabular}
            \caption{Router 2 Forwarding Table}
        \end{table}

        \begin{table}[!ht]
            \centering
            \begin{tabular}{|l|l|}
            \hline
                Longest Prefix Match & Outgoing Interface \\ \hline
                11010110 01100001 11111110 0 & B \\ \hline
                11010110 01100001 11111110 0000000 & D \\ \hline
                11010110 01100001 11111110 0000001  & E \\ \hline
            \end{tabular}
            \caption{Router 3 Forwarding Table}
        \end{table}

\end{enumerate}

\section*{Problem 3}    

\begin{enumerate}
    \item [(a)] 
        The header of each packet has 20 bytes, MTU = 700 bytes means for each packet we have header 20 + content 680 bytes.
        Then the 2400 bytes datagram = header 20 + content 2380.
        The link will cut the datagram in 680, 680, 680, 340 four fragments.

    \item[(b)]
        Identification: 1 1 1 1
        fragflag: 1 1 1 0
        fragment offset: 0 85 170 255
        length: 700 700 700 360

\end{enumerate}

\section*{Problem 4}

\begin{enumerate}
    \item [(a)] 
        We can record all packets sent from the NAT, since the hosts send the packet with sequencial numbers and a random initial ID number, we can group the consecutive ID number to a cluster.
        Then we can find the initial number of the cluster. The number of cluster means the number of distinct initial number, so we will know the number of host through the distinct initial ID number. 
    \item[(b)]
        The technique used in (a) is based on the sequencial number. Because of the sequencial assigned number, we can find the initial ID number in technique (a).
        If the ID number is randomly assigned, we cannot find the assigned rule of the ID number. Hence, we cannot know which number is the initial ID number of the host.
    
\end{enumerate}

\newpage

\section*{Problem 5}
Flow table for s2:
\begin{table}[!ht]
    \centering
    \begin{tabular}{|c|c|}
    \hline
        match & action \\ \hline
        \makecell{ingressport = 1 \\ IP Src = 10.3.0.* \\ IP Dst = 10.1.0.*} & forward port (2) \\ \hline
        \makecell{ingressport = 2 \\ IP src = 10.1.0.* \\ IP dst = 10.3.0.*} & forward port (1) \\ \hline
        IP dst = 10.2.0.3 & forward port (3) \\ \hline
        IP dst = 10.2.0.4 & forward port (4) \\ \hline
    \end{tabular}
\end{table}

\section*{Problem 6}
Dijkstra's algorithm table:
\begin{table}[!ht]
    \centering
    \begin{tabular}{|l|l|l|l|l|l|l|l|}
    \hline
        step & N' & D(t), p(t) & D(u), p(u) & D(v), p(v) & D(w), p(w) & D(y), p(y) & D(z),p(z) \\ \hline
        0 & x & $\infty$ & $\infty$ & 3,x & 6,x & 6,x & 8,x \\ \hline
        1 & xv & 7,v & 6,v & - & 6,x & 6,x & 8,x \\ \hline
        2 & xvw & 7,v & 6,v & - & - & 6,x & 8,x \\ \hline
        3 & xvwy & 7,v & 6,v & - & - & - & 8,x \\ \hline
        4 & xvwyv & 7,v & - & - & - & - & 8,x \\ \hline
        5 & xvwyvt & - & - & - & - & - & 8,x \\ \hline
        6 & xvwyzut & - & - & - & - & - & - \\ \hline
    \end{tabular}
    \caption{Dijkstra's algorithm table}
\end{table}
\section*{Problem 7}

\begin{enumerate}
    \item [(a)] 
        $D_{x}(w) = 2, D_{x}(y) = 4, D_{x}(u) = 7$
    \item[(b)]
        $D_{x}(u) = \min((c(x,w) + D_{w}(u)), (c(x,y) + D_{y}(u)))$ If $c(x,w)$ changes to 1, $D_{x}(u) = \min(1+5, 3+6) = 6 < 7$.
        Then x will inform its neighbors of a new minimum-cost path to u.
    \item[(c)]
        $D_{x}(u) = \min((c(x,w) + D_{w}(u)), (c(x,y) + D_{y}(u)))$ If $c(x,y)$ changes to 5, $D_{x}(u) = \min(2+5, 5+6) = 7 = 7$.
        Then x will not inform its neighbors of a new minimum-cost path to u since the distance-vector did not change.
        
\end{enumerate}

\newpage

\section*{Problem 8}

\begin{enumerate}
    \item [(a)] 
        eBGP
    \item[(b)]
        iBGP
    \item[(c)]
        eBGP
    \item[(d)]
        iBGP

\end{enumerate}

\end{document}
